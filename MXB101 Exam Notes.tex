%!TEX TS-program = xelatex
%!TEX options = -aux-directory=Debug -shell-escape -file-line-error -interaction=nonstopmode -halt-on-error -synctex=1 "%DOC%"
\documentclass{article}
\input{LaTeX-Submodule/template.tex}

% Additional packages & macros
\usepackage{changepage} % Modify page width
\usepackage{multicol} % Use multiple columns
\usepackage[explicit]{titlesec} % Modify section heading styles

\titleformat{\section}{\raggedright\normalfont\bfseries}{}{0em}{#1}
\titleformat{\subsection}{\raggedright\normalfont\small\bfseries}{}{0em}{#1}

%% A4 page
\geometry{
	a4paper,
	margin = 10mm
}

%% Hide horizontal rule 
\renewcommand{\headrulewidth}{0pt}
\fancyhead{}

%% Hide page numbers
\pagenumbering{gobble}

%% Multi-columns setup
\setlength\columnsep{4pt}

%% Paragraph setup
\setlength\parindent{0pt}
\setlength\parskip{0pt}

%% Customise section heading styles
% \titleformat*\section{\raggedright\bfseries}

\begin{document}
% Modify spacing
\titlespacing*\section{0pt}{1ex}{1ex}
\titlespacing*\subsection{0pt}{1ex}{1ex}
%
% \setlength\abovecaptionskip{0pt}
\setlength\belowcaptionskip{-15pt}
\setlength\textfloatsep{0pt}
%
\setlength\abovedisplayskip{1pt}
\setlength\belowdisplayskip{1pt}

\begin{multicols}{3}
    \section{Events and Probability}
    \subsection{Event}
    Set of outcomes from an experiment.
    \subsection{Sample Space}
    Set of all possible outcomes \(\Omega\).
    \subsection{Intersection}
    Outcomes occur in both \(A\) and \(B\)
    \begin{equation*}
        A \cap B \quad\quad \text{or} \quad\quad AB
    \end{equation*}
    \subsection{Disjoint}
    No common outcomes
    \begin{align*}
        AB                                & = \varnothing                                    \\
        \Pr{\left( AB \right)}            & = 0 \implies \Pr{\left( \varnothing \right)} = 0 \\
        \Pr{\left( A \,\vert\, B \right)} & = 0
    \end{align*}
    \subsection{Union}
    Set of outcomes in either \(A\) or \(B\)
    \begin{equation*}
        A \cup B
    \end{equation*}
    \subsection{Complement}
    Set of all outcomes not in \(A\), but in \(\Omega \)
    \begin{align*}
        A\overline{A}       & = \varnothing \\
        A \cup \overline{A} & = \Omega
    \end{align*}
    \subsection{Subset}
    \(A\) is a (non-strict) subset of \(B\) if all elements in \(A\) are also in \(B\) --- \(A \subset B\).
    \begin{equation*}
        AB = A \quad\quad \text{and} \quad\quad A \cup B = B
    \end{equation*}
    \begin{equation*}
        \forall A:A\subset \Omega \land \varnothing \subset A
    \end{equation*}
    \begin{align*}
        \Pr{\left( A \right)}             & \leq \Pr{\left( B \right)}                            \\
        \Pr{\left( B \,\vert\, A \right)} & = 1                                                   \\
        \Pr{\left( A \,\vert\, B \right)} & = \frac{\Pr{\left( A \right)}}{\Pr{\left( B \right)}}
    \end{align*}
    \subsection{Identities}
    \begin{align*}
        A \left( BC \right)            & = \left( AB \right) C                             \\
        A \cup \left( B \cup C \right) & = \left( A \cup B \right) \cup C                  \\
        A \left(B \cup C\right)        & = AB \cup AC                                      \\
        A \cup BC                      & = \left( A \cup B \right) \left( A \cup C \right)
    \end{align*}
    \subsection{Probability}
    Measure of the likeliness of an event occurring
    \begin{equation*}
        \Pr{\left( A \right)} \quad\quad \text{or} \quad\quad \mathrm{P}\left( A \right)
    \end{equation*}
    \begin{equation*}
        0 \leq \Pr{\left( E \right)} \leq 1
    \end{equation*}
    where a probability of 0 never happens, and 1 always happens.
    \begin{align*}
        \Pr{\left( \Omega \right)}       & = 1                         \\
        \Pr{\left( \overline{E} \right)} & = 1 - \Pr{\left( E \right)}
    \end{align*}
    \subsection{Multiplication Rule}
    For independent events \(A\) and \(B\)
    \begin{equation*}
        \Pr{\left( AB \right)} = \Pr{\left( A \right)} \Pr{\left( B \right)}.
    \end{equation*}
    For dependent events \(A\) and \(B\)
    \begin{equation*}
        \Pr{\left( AB \right)} = \Pr{\left( A \,\vert \, B \right)} \Pr{\left( B \right)}
    \end{equation*}
    \subsection{Addition Rule}
    For independent \(A\) and \(B\)
    \begin{equation*}
        \Pr{\left( A \cup B \right)} = \Pr{\left( A \right)} + \Pr{\left( B \right)} - \Pr{\left( AB \right)}.
    \end{equation*}
    If \(AB = \varnothing \), then \(\Pr{\left( AB \right)} = 0\), so that \(\Pr{\left( A \cup B \right)} = \Pr{\left( A \right)} + \Pr{\left( B \right)}\).
    \subsection{De Morgan's Laws}
    \begin{align*}
        \overline{A \cup B} & = \overline{A} \ \overline{B}     \\
        \overline{AB}       & = \overline{A} \cup \overline{B}.
    \end{align*}
    \begin{align*}
        \Pr{\left( A \cup B \right)} & = 1 - \Pr{\left( \overline{A} \ \overline{B} \right)}    \\
        \Pr{\left( AB \right)}       & = 1 - \Pr{\left( \overline{A} \cup \overline{B} \right)}
    \end{align*}
    \subsection{Circuits}
    A signal can pass through a circuit if there is a functional path from start to finish where
    each component functions independently.

    Let \(W_i\) be the event where component \(i\) functions 
    and \(S\) be the event where the system functions, then
    \begin{equation*}
        \Pr{\left( W_i \right)} = p
    \end{equation*}
    and \(\Pr{\left( S \right)}\) will be a function of \(p\) defined \(f:\left[ 0,\; 1 \right] \to \left[ 0,\; 1 \right]\).
    \subsection{Conditional Probability}
    The probability of event \(A\) given \(B\) has already occurred
    \begin{equation*}
        \Pr{\left( A \,\vert\, B \right)} = \frac{\Pr{\left( A B \right)}}{\Pr{\left( B \right)}}.
    \end{equation*}
    \(A\) and \(B\) are independent events if
    \begin{align*}
        \Pr{\left( A \,\vert\, B \right)} & = \Pr{\left( A \right)} \\
        \Pr{\left( B \,\vert\, A \right)} & = \Pr{\left( B \right)}
    \end{align*}
    the following statements are also true
    \begin{align*}
        \Pr{\left( A \,\vert\, \overline{B} \right)}            & = \Pr{\left( A \right)}            \\
        \Pr{\left( \overline{A} \,\vert\, B \right)}            & = \Pr{\left( \overline{A} \right)} \\
        \Pr{\left( \overline{A} \,\vert\, \overline{B} \right)} & = \Pr{\left( \overline{A} \right)}
    \end{align*}
    \subsection{Probability Rules with Conditional}
    \begin{align*}
        \Pr{\left( \overline{A} \,\vert\, C \right)} & = 1 - \Pr{\left( A \,\vert\, C \right)}                                 \\
        \Pr{\left( A \cup B \,\vert\, C \right)}     & = \begin{aligned}[t]
                                                             \Pr{\left( A \,\vert\, C \right)} + \Pr{\left( B \,\vert\, C \right)} \\
                                                             - \Pr{\left( AB \,\vert\, C \right)}
                                                         \end{aligned} \\
        \Pr{\left( A B \,\vert\, C \right)}          & = \Pr{\left( A \,\vert\, BC \right)} \Pr{\left( B \,\vert\, C \right)}
    \end{align*}
    \subsection{Conditional Independence}
    Given \(\Pr{\left( A \,\vert\, B \right)} \neq \Pr{\left( A \right)}\)
    \(A\) and \(B\) are conditionally dependent given \(C\) if
    \begin{equation*}
        \Pr{\left( A \,\vert\, BC \right)} = \Pr{\left( A \,\vert\, C \right)}.
    \end{equation*}
    Futhermore
    \begin{equation*}
        \Pr{\left( AB \,\vert\, C \right)} = \Pr{\left( A \,\vert\, C \right)} \Pr{\left( B \,\vert\, C \right)}.
    \end{equation*}
    Conversely
    \begin{align*}
        \Pr{\left( A \,\vert\, B \right)}  & = \Pr{\left( A \right)}                                                \\
        \Pr{\left( A \,\vert\, BC \right)} & \neq \Pr{\left( A \,\vert\, C \right)}                                 \\
        \Pr{\left( AB \,\vert\, C \right)} & = \Pr{\left( A \,\vert\, BC \right)} \Pr{\left( B \,\vert\, C \right)}
    \end{align*}
    Pairwise independence does not imply mutual independence for three events.
    % \begin{gather*}
    %     \begin{cases}
    %         \Pr{\left( A B \right)} = \Pr{\left( A \right)} \Pr{\left( B \right)} \\
    %         \Pr{\left( A C \right)} = \Pr{\left( A \right)} \Pr{\left( C \right)} \\
    %         \Pr{\left( B C \right)} = \Pr{\left( B \right)} \Pr{\left( C \right)}
    %     \end{cases} \not\Rightarrow                                                                          \\
    %     \Pr{\left( A B C \right)} = \Pr{\left( A \right)} \Pr{\left( B \right)} \Pr{\left( C \right)}.
    % \end{gather*}
    Independence should not be assumed unless explicitly stated.
    \subsection{Marginal Probability}
    The probability of an event irrespective of the outcome of another variable.
    \subsection{Total Probability}
    \(A = AB \cup A\overline{B}\)
    \begin{align*}
        \Pr{\left( A \right)} & = \Pr{\left( AB \right)} + \Pr{\left( A\overline{B} \right)} \\
        \Pr{\left( A \right)} & = \begin{aligned}[t]
                                      &\Pr{\left( A \,\vert\, B \right)}\Pr{\left( B \right)} \\
                                      &+ \Pr{\left( A \,\vert\, \overline{B} \right)}\Pr{\left( \overline{B} \right)}
                                  \end{aligned}
    \end{align*}
    In general, partition \(\Omega\) into disjoint events \(B_1,\; B_2,\; \dots,\; B_n\),
    such that \(\bigcup_{i=1}^n B_i = \Omega\)
    \begin{equation*}
        \Pr{\left( A \right)} = \sum_{i = 1}^n \Pr{\left( A \,\vert\, B_i \right)}\Pr{\left( B_i \right)}
    \end{equation*}
    \subsection{Bayes' Theorem}
    \begin{equation*}
        \Pr{\left( A \,\vert\, B \right)} = \frac{\Pr{\left( B \,\vert\, A \right)}\Pr{\left( A \right)}}{\Pr{\left( B \right)}}
    \end{equation*}
    \section{Combinatorics}
    \subsection{Number of Outcomes}
    Let \(\abs{A}\) denote the number of outcomes in an event \(A\).

    For \(k\) disjoint events \({\left\{ S_1,\:\ldots,\:S_k \right\}}\)
    where the \(i\)th event has \(n_i\) possible outcomes,
    \subsection{Addition Principle}
    Number of possible samples from any event
    \begin{equation*}
        \abs*{\bigcup_{i = 0}^{k} S_i} = \sum_{i = 1}^k n_i
    \end{equation*}
    \subsection{Multiplication Principle}
    Number of possible samples from every event
    \begin{equation*}
        \abs*{\bigcap_{i=0}^{k} S_i} = \prod_{i = 1}^k n_i
    \end{equation*}
    \subsection{Counting Probability}
    If \(S_i\) has equally likely outcomes
    \begin{equation*}
        \Pr{\left( S_i \right)} = \frac{\abs{S_i}}{\abs{S}}
    \end{equation*}
    \subsection{Ordered Sampling with Replacement}
    Number of ways to choose \(k\) objects from a set with \(n\) elements
    \begin{equation*}
        n^k
    \end{equation*}
    \subsection{Ordered Sampling without Replacement}
    Number of ways to arrange \(k\) objects from a set of \(n\) elements,
    or the \(k\)-permutation of \(n\)-elements
    \begin{align*}
        \prescript{n}{}{P}_k & = \frac{n!}{\left( n - k \right)!}
    \end{align*}
    for \(0 \leq k \leq n\).

    An \(n\)-permutation of \(n\) elements is the permutation of those elements.
    \begin{align*}
        \prescript{n}{}{P}_n & = n!
    \end{align*}
    \subsection{Unordered Sampling without Replacement}
    Number of ways to choose \(k\) objects from a set of \(n\) elements,
    or the \(k\)-combination of \(n\)-elements
    \begin{align*}
        \prescript{n}{}{C}_k = \frac{\prescript{n}{}{P}_k}{k!} = \frac{n!}{k! \left( n - k \right)!}
    \end{align*}
    for \(0 \leq k \leq n\).
    \subsection{Unordered Sampling with Replacement}
    Number of ways to choose \(k\) objects from a set with \(n\) elements
    \begin{equation*}
        \binom{n + k - 1}{k}
    \end{equation*}
\end{multicols}
\begin{table}[H]
    \centering
    \begin{tabular}{c c c c c c}
        \toprule
        \textbf{Distribution}                                      & \textbf{Restrictions}                                  & \textbf{PMF}                                              & \textbf{CDF}                                                             & \(\E{\left( X \right)}\)            & \(\Var{\left( X \right)}\)                    \\
        \midrule
        \(X \sim \operatorname{Uniform}{\left( a,\: b \right)}\)   & \(x \in \left\{ a, \dots, b \right\}\)                 & \(\frac{1}{b - a + 1}\)                                   & \(\frac{x - a + 1}{b - a + 1}\)                                          & \(\frac{a + b}{2}\)                 & \(\frac{\left( b - a + 1 \right)^2 - 1}{12}\) \\
        \(X \sim \operatorname{Bernoulli}{\left( p \right)}\)      & \(p \in \interval{0}{1}, x \in \left\{ 0, 1 \right\}\) & \(p^x \left( 1 - p \right)^{1 - x}\)                      & \(1 - p\)                                                                & \(p\)                               & \(p \left( 1 - p \right)\)                    \\
        \(X \sim \operatorname{Binomial}{\left( n,\: p \right)}\)  & \(x \in \left\{ 0, \dots, n \right\}\)                 & \(\binom{n}{x} p^x \left( 1 - p \right)^{n - x}\)         & \(\sum_{u = 0}^x \binom{n}{u} p^u \left( 1 - p \right)^{n - u}\)         & \(np\)                              & \(np\left( 1 - p \right)\)                    \\
        \(N \sim \operatorname{Geometric}{\left( p \right)}\)      & \(n \geq 1\)                                           & \(\left( 1 - p \right)^{n - 1} p\)                        & \(1 - \left( 1 - p \right)^n\)                                           & \(\frac{1}{p}\)                     & \(\frac{1 - p}{p^2}\)                         \\
        \( Y \sim \operatorname{Geometric}{\left( p \right)}\)     & \(y \geq 0\)                                           & \(\left( 1 - p \right)^y p\)                              & \(1 - \left( 1 - p \right)^{y + 1}\)                                     & \(\frac{1 - p}{p}\)                 & \(\frac{1 - p}{p^2}\)                         \\
        \( N \sim \operatorname{NB}{\left( k,\: p \right)}\)       & \(n \geq k\)                                           & \(\binom{n - 1}{k - 1} \left( 1 - p \right)^{n - k} p^k\) & \(\sum_{u = k}^n \binom{u - 1}{k - 1} \left( 1 - p \right)^{u - k} p^k\) & \(\frac{k}{p}\)                     & \(\frac{k\left( 1 - p \right)}{p^2}\)         \\
        \( Y \sim \operatorname{NB}{\left( k,\: p \right)}\)       & \(y \geq 0\)                                           & \(\binom{y + k - 1}{k - 1} \left( 1 - p \right)^y p^k\)   & \(\sum_{u = 0}^y \binom{u + k - 1}{k - 1} \left( 1 - p \right)^u p^k\)   & \(\frac{k\left( 1 - p \right)}{p}\) & \(\frac{k\left( 1 - p \right)}{p^2}\)         \\
        \( N \sim \operatorname{Poisson}{\left( \lambda \right)}\) & \(n \geq 0\)                                           & \(\frac{\lambda^n e^{-\lambda}}{n!}\)                     & \(e^{-\lambda} \sum_{u = 0}^n \frac{\lambda^u}{u!}\)                     & \(\lambda\)                         & \(\lambda\)                                   \\
        \bottomrule
    \end{tabular}
    \caption{Discrete probability distributions.} % \label{}
\end{table}
\begin{table}[H]
    \centering
    \begin{tabular}{c c c c c c}
        \toprule
        \textbf{Distribution}                                       & \textbf{Restrictions}                  & \textbf{PMF}                                                                         & \textbf{CDF}                                                                            & \(\E{\left( X \right)}\) & \(\Var{\left( X \right)}\)            \\
        \midrule
        \(X \sim \operatorname{Uniform}{\left( a,\: b \right)}\)    & \(a < x < b\)                          & \(\frac{1}{b - a}\)                                                                  & \(\frac{x - a}{b - a}\)                                                                 & \(\frac{a + b}{2}\)      & \(\frac{\left( b - a \right)^2}{12}\) \\
        \(T \sim \operatorname{Exp}{\left( \eta \right)}\)          & \(t > 0\)                              & \(\eta e^{-\eta t}\)                                                                 & \(1 - e^{-\eta t}\)                                                                     & \(\frac{1}{\eta}\)       & \(p \left( 1 - p \right)\)            \\
        \(X \sim \operatorname{N}{\left( \mu,\: \sigma^2 \right)}\) & \(x \in \left\{ 0, \dots, n \right\}\) & \(\frac{1}{\sqrt{2 \pi \sigma^2}} e^{-\frac{\left( x - \mu \right)^2}{2 \sigma^2}}\) & \(\frac{1}{2} \left( 1 + \erf{\left( \frac{x - \mu}{\sigma \sqrt{2}} \right)} \right)\) & \(\mu\)                  & \(\sigma^2\)                          \\
        \bottomrule
    \end{tabular}
    \caption{Continuous probability distributions.} % \label{}
\end{table}
\begin{minipage}{126.1962963mm}
    \begin{table}[H]
        \centering
        \begin{tabular}{c c c }
            \toprule
                                       & \textbf{Discrete}                              & \textbf{Continuous}                                    \\
            \midrule
            Valid probabilities        & \(0 \leq p_x \leq 1\)                          & \(f(x) \geq 0\)                                        \\
            Cumulative probability     & \(\sum_{u \leq x} p_u\)                        & \(\int_{-\infty}^{x} f(u) \odif{u}\)                   \\
            \(\E{\left( X \right)}\)   & \(\sum_{\Omega} xp_x\)                         & \(\int_{\Omega} xf(x)\odif{x}\)                        \\
            \(\Var{\left( X \right)}\) & \(\sum_{\Omega} \left( x - \mu \right)^2 p_x\) & \(\int_{\Omega} \left( x - \mu \right)^2f(x)\odif{x}\) \\
            \bottomrule
        \end{tabular}
        \caption{Probability rules for univariate \(X\).} % \label{}
    \end{table}
    \begin{multicols}{2}
        \section{Random Variables}
        Measurable variable whose value holds some uncertainty.
        An event is when a random variable assumes a certain value or range of values.
        \subsection{Probability distribution}
        The probability distribution of a random variable \(X\) is a function that links all outcomes \(x \in \Omega\)
        to the probability that they will occur \(\Pr{\left( X = x \right)}\).
        \subsection{Probability mass function}
        \begin{equation*}
            \Pr{\left( X = x \right)} = p_x
        \end{equation*}
        \subsection{Probability density function}
        \begin{equation*}
            \Pr{\left( x_1 \leq X \leq x_2 \right)} = \int_{x_1}^{x_2} f\left( x \right) \odif{x}
        \end{equation*}
        \subsection{Cumulative distribution function}
        Probability that a random variable is
        less than or equal to a particular realisation \(x\).

        \(F\left( x \right)\) is a valid CDF if:
        \begin{enumerate}
            \item \(F\) is monotonically increasing and continuous
            \item \(\lim_{x \to -\infty} F\left( x \right) = 0\)
            \item \(\lim_{x \to \infty} F\left( x \right) = 1\)
        \end{enumerate}
        \begin{equation*}
            \odv{F\left( x \right)}{x} = \odv{}{x} \int_{-\infty}^x f\left( u \right) \odif{u} = f\left( x \right)
        \end{equation*}
        \subsection{Complementary CDF (survival)}
        \begin{equation*}
            \Pr{\left( X > x \right)} = 1 - \Pr{\left( X \leq x \right)} = 1 - F\left( x \right)
        \end{equation*}
        \subsection{\texorpdfstring{\(p\)}{p}-Quantile}
        \begin{equation*}
            F\left( x \right) = \int_{-\infty}^x f\left( u \right) \odif{u} = p
        \end{equation*}
        \subsection{Median}
        \begin{equation*}
            \int_{-\infty}^m f\left( u \right) \odif{u} = \int^{\infty}_m f\left( u \right) \odif{u} = \frac{1}{2}
        \end{equation*}
        \subsection{Lower and upper quartile}
        \begin{equation*}
            \int_{-\infty}^{q_1} f\left( u \right) \odif{u} = \frac{1}{4}
        \end{equation*}
        and
        \begin{equation*}
            \int_{-\infty}^{q_2} f\left( u \right) \odif{u} = \frac{3}{4}
        \end{equation*}
        \subsection{Quantile function}
        \begin{equation*}
            x = F^{-1}\left( p \right) = Q\left( p \right)
        \end{equation*}
        \subsection{Expectation (mean)}
        Expected value given an infinite number of observations. For \(a < c < b\):
        \begin{equation*}
            \E{\left(X\right)} = \begin{aligned}[t]
                 & -\int_{a}^c F\left( x \right) \odif{x}                     \\
                 & + \int_c^b \left(1 - F\left( x \right)\right) \odif{x} + c
            \end{aligned}
        \end{equation*}
        \subsection{Variance}
        Measure of spread of the distribution (average squared distance of each value from the mean).
        \begin{equation*}
            \Var{\left( X \right)} = \sigma^2 = \E{\left( X^2 \right)} - \E{\left( X \right)}^2
        \end{equation*}
        \subsection{Standard deviation}
        \begin{equation*}
            \sigma = \sqrt{\Var{\left( X \right)}}
        \end{equation*}
    \end{multicols}
\end{minipage}\hfill%
\begin{minipage}{62.39259259mm}
    \subsection{Uniform Distribution}
    Single trial \(X\) in a set of equally likely elements.
    \subsection{Bernoulli (binary) Distribution}
    Boolean-valued outcome \(X\), i.e., success (1) or failure (0).
    \(\left( 1 - p \right)\) is sometimes denoted as \(q\).
    \subsection{Binomial Distribution}
    Number of successes \(X\) for \(n\) independent trials with the same probability of success \(p\).
    \begin{align*}
        X   & = Y_1 + \cdots + Y_n                                                                                                           \\
        Y_i & \overset{\mathrm{iid}}{\sim} \operatorname{Bernoulli}{\left( p \right)} : \forall i \in \left\{ 1,\: 2,\: \dots,\: n \right\}.
    \end{align*}
    \subsection{Geometric Distribution}
    Number of trials \(N\) up to and including the first success where each trial is independent and has the same probability of success \(p\).
    \subsection{Alternate Geometric}
    Number of failures \(Y = N - 1\) until a success.
    \subsection{Negative Binomial Distribution}
    Number of trials \(N\) until \(k \geq 1\) successes, where each trial is independent and has the same probability of success \(p\).
    \begin{align*}
        N   & = Y_1 + Y_2 + \cdots + Y_k                                                                                                \\
        Y_i & \overset{\mathrm{iid}}{\sim} \operatorname{Geom}{\left( p \right)} : \forall i \in \left\{ 1,\: 2,\: \dots,\: k \right\}.
    \end{align*}
    \subsection{Alternate Negative Binomial}
    Number of failures \(Y = N - k\) until \(k\) successes:
    \subsection{Poisson Distribution}
    Number of events \(N\) which occur over a fixed interval of time \(\lambda\).
    \subsection{Modelling Count Data}
    \begin{itemize}
        \setlength\itemsep{-0.2em}
        \item Poisson (mean = variance)
        \item Binomial (underdispersed, mean > variance)
        \item Geometric/Negative Binomial \newline (overdispersed, mean < variance)
    \end{itemize}
\end{minipage}
\begin{minipage}{62.39259259mm}
    \subsection{Uniform Distribution}
    Outcome \(X\) within some interval, where the probability of an outcome in one interval is the same as all other intervals of the same length.
    \begin{equation*}
        m = \frac{a + b}{2}
    \end{equation*}
    \subsection{Exponential Distribution}
    Time \(T\) between events with rate \(\eta\).
    \begin{equation*}
        m = \frac{\ln{\left( 2 \right)}}{\eta}
    \end{equation*}
    \subsection{Memoryless Property}
    For \(T \sim \operatorname{Exp}{\left( \lambda \right)}\):
    \begin{equation*}
        \Pr{\left( T > s + t \,\vert\, T > t \right)} = \Pr{\left( T > s \right)}
    \end{equation*}
    For \(N \sim \operatorname{Geometric}{\left( p \right)}\):
    \begin{equation*}
        \Pr{\left( N > s + n \,\vert\, N > n \right)} = \Pr{\left( N > s \right)}
    \end{equation*}
    \subsection{Normal Distribution}
    Used to represent random situations, i.e., measurements and their errors.
    Also used to approximate other distributions.
    \subsection{Standard Normal Distribution}
    Given \(X \sim \operatorname{N}{\left( \mu,\: \sigma^2 \right)}\), consider
    \begin{equation*}
        Z = \frac{X - \mu}{\sigma}
    \end{equation*}
    so that \(Z \sim \operatorname{N}{\left( 0,\: 1 \right)}\).
\end{minipage}\hfill%
\begin{minipage}{126.1962963mm}
    \begin{multicols}{2}
        \section{Central Limit Theorem}
        The sum of independent and identically distributed random variables, when properly standardised,
        can be approximated by a normal distribution, as \(n \to \infty\).

        Let \(X_1,\: \ldots,\: X_n \overset{\mathrm{iid}}{\sim} X\) with
        \(\E{\left( X \right)} = \mu\) and \(\Var{\left( X \right)} = \sigma^2\):
        \subsection{Average of Random Variables}
        If \(\overline{X} = \frac{1}{n} \sum_{i = 1}^n X_i\):
        \begin{align*}
            \E{\left( \overline{X} \right)}   & = \mu                \\
            \Var{\left( \overline{X} \right)} & = \frac{\sigma^2}{n}
        \end{align*}
        By standardising \(\overline{X}\), we can define
        \begin{equation*}
            Z = \lim_{n \to \infty} \frac{\overline{X} - \mu}{\sigma / \sqrt{n}}
        \end{equation*}
        so that \(Z \to \operatorname{N}{\left( 0,\: 1 \right)}\) as \(n \to \infty\).
        \subsection{Sum of Random Variables}
        If \(\overline{Y} = \sum_{i = 1}^n X_i\):
        \begin{align*}
            \E{\left( Y \right)}   & = n \mu      \\
            \Var{\left( Y \right)} & = n \sigma^2
        \end{align*}
        \begin{equation*}
            Y \sim \operatorname{N}{\left( n \mu,\: n \sigma^2 \right)}
        \end{equation*}
        as \(n \to \infty\).
        \subsection{Binomial Approximations}
        If \(X \sim \operatorname{Binomial}{\left( n,\: p \right)}\):
        \begin{equation*}
            X \approx Y \sim \operatorname{N}{\left( np,\: np\left( 1 - p \right) \right)}
        \end{equation*}
        Sufficient for \(np > 5\) and \(n\left( 1 - p \right) > 5\).

        If \(np < 5\):
        \begin{equation*}
            X \approx Y \sim \operatorname{Pois}{\left( np \right)}.
        \end{equation*}
        If \(n\left( 1 - p \right) < 5\), consider the number of failures \(W = n - X\):
        \begin{equation*}
            W \approx Y \sim \operatorname{Pois}{\left( n\left( 1 - p \right) \right)}.
        \end{equation*}
        \subsection{Continuity Correction}
        \begin{equation*}
            \Pr{\left( X \leq x \right)} = \Pr{\left( X < x + 1 \right)}
        \end{equation*}
        must hold for any \(x\). Therefore
        \begin{equation*}
            \Pr{\left( X \leq x \right)} \approx \Pr{\left( Y \leq x + \frac{1}{2} \right)}.
        \end{equation*}
        \subsection{Poisson Approximation}
        If \(X_i \sim \operatorname{Poisson}{\left( \lambda \right)}\):

        Let \(X = \sum_{i = 1}^n X_i\):
        \begin{align*}
            \E{\left( X \right)}   & = n \lambda \\
            \Var{\left( X \right)} & = n \lambda
        \end{align*}
        \begin{equation*}
            X \approx Y \sim \operatorname{N}{\left( n\lambda,\: n\lambda \right)}.
        \end{equation*}
        Sufficient for \(n \lambda > 10\), and for accurate approximations, \(n \lambda > 20\).
    \end{multicols}
\end{minipage}
\hrule
\begin{multicols}{3}
    \section{Bivariate Distributions}
    \subsection{Bivariate probability mass function}
    Distribution over the joint space of two discrete random variables \(X\) and \(Y\):
    \begin{align*}
        \Pr{\left( X = x,\: Y = y \right)}                                             & = p_{x,\: y} \geq 0 \\
        \sum_{y \in \Omega_2} \sum_{x \in \Omega_1} \Pr{\left( X = x,\: Y = y \right)} & = 1
    \end{align*}
    for all pairs of \(x \in \Omega_1\) and \(y \in \Omega_2\).
    The joint probability mass function can be shown using a table:
    {\small
    \begin{equation*}
        \begin{matrix}[c|ccc] % chktex 44
                   & y_1        & \cdots & y_n        \\
            \hline % chktex 44
            x_1    & p_{1,\: 1} & \cdots & p_{1,\: n} \\
            \vdots & \vdots     & \ddots & \vdots     \\
            x_n    & p_{n,\: 1} & \cdots & p_{n,\: n}
        \end{matrix}
    \end{equation*}
    }
    \subsection{Bivariate probability density function}
    Distribution over the joint space of two continuous random variables \(X\) and \(Y\):
    \begin{align*}
        \Pr{\left( x_1 \leq X \leq x_2,\: y_1 \leq Y \leq y_2 \right)} \\
        = \int_{x_1}^{x_2} \int_{y_1}^{y_2} f\left( x,\: y \right) \odif{y} \odif{x}
    \end{align*}
    This function must satisfy
    \begin{align*}
        f\left( x,\: y \right)                                                               & \geq 0 \\
        \int_{x \in \Omega_1} \int_{y \in \Omega_2} f\left( x,\: y \right) \odif{y} \odif{x} & = 1.
    \end{align*}
    for all pairs of \(x \in \Omega_1\) and \(y \in \Omega_2\).
    \begin{flalign*}
        \Pr{\left( X = x ,\: Y = y \right)} = \\
        \Pr{\left( X = x \,\vert\, Y = y \right)} \Pr{\left( Y = y \right)}
    \end{flalign*}
    \subsection{Marginal Probability}
    Probability function of each random variable.
    Must specify the range of values that variable can take.
    \subsection{Marginal probability mass function}
    \begin{align*}
        p_x & = \sum_{y \in \Omega_2} \Pr{\left( X = x,\: Y = y \right)} \\
        p_y & = \sum_{x \in \Omega_1} \Pr{\left( X = x,\: Y = y \right)}
    \end{align*}
    \subsection{Marginal probability density function}
    \begin{align*}
        f\left( x \right) & = \int_{y_1}^{y_2} f\left( x,\: y \right) \odif{y} \\
        f\left( y \right) & = \int_{x_1}^{x_2} f\left( x,\: y \right) \odif{x}
    \end{align*}
    \subsection{Conditional probability mass function}
    \begin{gather*}
        \Pr{\left( X = x \,\vert\, Y = y \right)} = \frac{\Pr{\left( X = x,\: Y = y \right)}}{\Pr{\left( Y = y \right)}} \\
        \sum_{x \in \Omega_1} \Pr{\left( X = x \,\vert\, Y = y \right)} = 1
    \end{gather*}
    \subsection{Conditional probability density function}
    \begin{gather*}
        f\left( x \,\vert\, y \right) = \frac{f\left( x,\: y \right)}{f\left( y \right)} \\
        \int_{x_1}^{x_2} f\left( x \,\vert\, y \right) \odif{x} = 1
    \end{gather*}
    \subsection{Independence}
    Two discrete random variables \(X\) and \(Y\) are independent if
    \begin{equation*}
        \Pr{\left( X = x \,\vert\, Y = y \right)} = \Pr{\left( X = x \right)}
    \end{equation*}
    for all pairs of \(x\) and \(y\).

    Two continuous random variables \(X\) and \(Y\) are independent if
    \begin{equation*}
        f\left( x,\: y \right) \propto g\left( x \right) h\left( y \right)
    \end{equation*}
    so that
    \begin{equation*}
        f\left( x \,\vert\, y \right) = f\left( x \right).
    \end{equation*}
    \subsection{Conditional Expectation}
    \begin{align*}
        \E{\left( X \,\vert\, Y = y \right)} & = \sum_{x\in\Omega_1} x p_{x\,\vert\,y}                     \\
        \E{\left( X \,\vert\, Y = y \right)} & = \int_{x_1}^{x_2} x f\left( x \,\vert\, y \right) \odif{x}
    \end{align*}
    \subsection{Conditional Variance}
    \begin{flalign*}
        \Var{\left( X \,\vert\, Y = y \right)} \\
        = \E{\left( X^2 \,\vert\, Y = y \right)} - \E{\left( X \,\vert\, Y = y \right)}^2
    \end{flalign*}
    \subsection{Law of Total Expectation}
    By treating \(\E{\left( X \,\vert\, Y \right)}\) as a random variable of \(Y\):
    \begin{equation*}
        \E{\left( X \right)} = \E{\left( \E{\left( X \,\vert\, Y \right)} \right)}
    \end{equation*}
    \subsection{Joint expectation}
    \begin{align*}
        \E{\left( XY \right)} & = \sum_{x\in\Omega_1} \sum_{y\in\Omega_2} xy p_{x,\: y}                          \\
        \E{\left( XY \right)} & = \int_{x_1}^{x_2} \int_{x_1}^{x_2} xy f\left( x,\: y \right) \odif{y} \odif{x}.
    \end{align*}
    \subsection{Transformation rules}
    \begin{align*}
        \E{\left( aX \pm b \right)}   & = a\E{\left( X \right)} \pm b                        \\
        \E{\left( X \pm Y \right)}    & = \E{\left( X \right)} \pm \E{\left( Y \right)}      \\
        \Var{\left( aX \pm b \right)} & = a^2\Var{\left( X \right)}                          \\
        \Var{\left( X \pm Y \right)}  & = \begin{aligned}[t]
                                               & \Var{\left( X \right)} + \Var{\left( Y \right)} \\
                                               & \pm 2\Cov{\left( X,\: Y \right)}
                                          \end{aligned}
    \end{align*}
    \begin{equation*}
        \Cov{\left( aX + b,\: cY + d \right)} = ac \Cov{\left( X,\: Y \right)}
    \end{equation*}
    If \(X\) and \(Y\) are independent:
    \begingroup
    \allowdisplaybreaks
    \begin{align*}
        \E{\left( X \,\vert\, Y = y \right)}   & = \E{\left( X \right)}                            \\
        \Var{\left( X \,\vert\, Y = y \right)} & = \Var{\left( X \right)}                          \\
        \Var{\left( X \pm Y \right)}           & = \Var{\left( X \right)} + \Var{\left( Y \right)} \\
        \E{\left( XY \right)}                  & = \E{\left( X \right)} \E{\left( Y \right)}
    \end{align*}
    \endgroup
    \begin{align*}
        \Var{\left( XY \right)}      = \Var{\left( X \right)} \Var{\left( Y \right)} \\
        + \E{\left( X \right)}^2 \Var{\left( Y \right)} + \E{\left( Y \right)}^2 \Var{\left( X \right)}
    \end{align*}
    for constants \(a\), \(b\), \(c\), and \(d\).
    \subsection{Covariance}
    Measure of the dependence between two random variables
    \begin{align*}
        \Cov{\left( X,\: Y \right)} & = \begin{aligned}[t]
                                            \E\left( \left( X - \E{\left( X \right)} \right) \right. & \\
                                            \left. \left( Y - \E{\left( Y \right)} \right) \right)   &
                                        \end{aligned}                        \\
                                    & = \E{\left( XY \right)} - \E{\left( X \right)} \E{\left( Y \right)}
    \end{align*}
    The covariance of \(X\) and \(Y\) is:
    \begin{description}
        \item[Positive] if an increase in one variable is more likely to result in an increase in
            the other variable.
        \item[Negative] if an increase in one variable is more likely to result in a decrease in
            the other variable.
        \item[Zero] if \(X\) and \(Y\) are independent. Note that the converse is not true.
    \end{description}
    Describes the direction of a relationship, but does not quantify the strength of such a relationship.
    \subsection{Correlation}
    Explains both the direction and strength of a linear relationship between two random variables.
    \begin{equation*}
        \rho\left( X,\: Y \right) = \frac{\Cov{\left( X,\: Y \right)}}{\sqrt{\Var{\left( X \right)} \Var{\left( Y \right)}}}
    \end{equation*}
    where \(-1 \leq \rho\left( X,\: Y \right) \leq 1\).

    The correlation is interpretted as follows:
    \begin{itemize}
        \item \(\rho\left( X,\: Y \right) > 0\) iff \(X\) and \(Y\) have a positive linear relationship.
        \item \(\rho\left( X,\: Y \right) < 0\) iff \(X\) and \(Y\) have a negative linear relationship.
        \item \(\rho\left( X,\: Y \right) = 0\) if \(X\) and \(Y\) are independent. Note that the converse is not true.
        \item \(\rho\left( X,\: Y \right) = 1\) iff \(X\) and \(Y\) have a perfect linear relationship with positive slope.
        \item \(\rho\left( X,\: Y \right) = -1\) iff \(X\) and \(Y\) have a perfect linear relationship with negative slope.
    \end{itemize}
    The slope of a perfect linear relationship cannot be obtained from the correlation.
\end{multicols}
\hrule
\begin{multicols}{3}
    \section{Markov Chains}
    A Markov chain is a discrete time and state stochastic process that describes how a state evolves over time.
    %     In this process, the set of all states is discrete and disjoint and states change probabilistically so that
    %     a step may not result in a changed state.
    %     At each step, the next state depends only on the current state of the random variable.
    States are denoted by the random variable \(X_t\) at time step \(t\).
    \subsection{Markov Property}
    \begin{flalign*}
        \Pr{\left( X_t = x_t \,\vert\, X_{t-1} = x_{t-1},\: \ldots,\: X_{0} = x_{0} \right)} \\
        = \Pr{\left( X_t = x_t \,\vert\, X_{t-1} = x_{t-1} \right)}
    \end{flalign*}
    \subsection{Homogeneous Markov Chains}
    A Markov chain is homogeneous when
    \begin{flalign*}
        \Pr{\left( X_{t+n} = j \,\vert\, X_t = i \right)} = \\
        \Pr{\left( X_n = j \,\vert\, X_0 = i \right)} = p_{ij}^{(n)}
    \end{flalign*}
    %     that is, the \(n\)-step conditional probabilities do not depend on the time step \(t\).
    \subsection{Transition Probability Matrix}
    A homogeneous Markov chain is characterised by the transition probability matrix \(\symbf{P} \in \R^{m \times m}\), where
    \(m\) is the number of states.
    \(\symbf{P}\) must fulfil the following properties:
    \begin{itemize}
        \item \(p_{i,\:j} = \Pr{\left( X_t = j \,\vert\, X_{t-1} = i \right)}\)
        \item \(p_{i,\:j} \geq 0 : \forall i,\: j\)
        \item \(\sum_{j = 1}^m p_{i,\:j} = 1 : \forall j\)
    \end{itemize}
    \(\symbf{P}\) has the following form
    \begin{equation*}
        \symbf{P} = \quad \scriptscriptstyle{X_t} \overset{X_{t+1}}{\begin{bmatrix}
                \phantom{p} & \phantom{p} \\
                \phantom{p} & \phantom{p}
            \end{bmatrix}}
    \end{equation*}
    The \(n\)-step transition probability is given by \(\symbf{P}^n\).
    \subsection{Unconditional State Probabilities}
    The unconditional probability of being in state \(j\) at time \(n\) is given by
    \begin{equation*}
        \Pr{\left( X_n = j \right)} = p_j^{(n)}
    \end{equation*}
    Given multiple states, let \(\symbfit{s}^{(n)}\) denote the vector of all states \(p_j^{(n)}\) at
    time \(n\). Then
    \begin{align*}
        {\symbfit{s}^{(n)}}^\top & = {\symbfit{s}^{(n-1)}}^\top \symbf{P} \\
        {\symbfit{s}^{(n)}}^\top & = {\symbfit{s}^{(0)}}^\top \symbf{P}^n
    \end{align*}
    \subsection{Stationary Distribution}
    At steady-state, the probability of being in a particular state does not change from one step to the next.
    \begin{equation*}
        \symbfit{s}^{(n+1)} = \symbfit{s}^{(n)} \implies {\symbfit{s}^{(n)}}^\top = {\symbfit{s}^{(n)}}^\top \symbf{P}
    \end{equation*}
    The stationary distribution \(\symbfit{\pi}\) satisfies \(\symbfit{\pi}^\top = \symbfit{\pi}^\top \symbf{P}\).
    To determine \(\symbfit{\pi}\), we must use the equation \(\sum_{i = 1}^m \pi_i = 1\).
    %     \subsection{Limiting Distribution}
    %     Under certain conditions, as \(n \to \infty\), each row of \(\symbf{P}^n\) will be equal to \(\symbfit{\pi}^\top\),
    %     so that each state moves to the next step with the same probability. This is known as the limiting distribution.
    %     Here \(\symbfit{\pi}\) provides the long run probabilities of being in each state and the process forgets where it starts.

    %     A sufficient condition for the above is if \(\symbf{P}^n\) has positive entries for some finite \(n\).

    %     \textit{Note that a stationary distribution does not imply that a limiting distribution exists}.
    \section{Poisson Processes}
    A Poisson process is a continuous time and discrete state stochastic process that counts events
    that occur randomly in time (or space).

    The rate parameter \(\eta\) is the average rate at which events occur.
    The rate does not depend on how long the process has been run nor how many events have already been
    observed.

    The number of events that occur randomly on the interval \(\ointerval{0}{t}\), are denoted by the random variable \(X\left( t \right)\).
    \begin{equation*}
        \Pr{\left( X\left( 0 \right) = 0 \right)} = 1.
    \end{equation*}
    Let \(h\) be a small interval such that at most 1 event can occur during that time, then

    \begin{gather*}
        \Pr{\left( X\left( t + h \right) = n + 1 \,\vert\, X\left( t \right) = n \right)} \approx \eta h     \\
        \Pr{\left( X\left( t + h \right) = n \,\vert\, X\left( t \right) = n \right)}     \approx 1 - \eta h \\
        \Pr{\left( X\left( t + h \right) > n + 1 \,\vert\, X\left( t \right) = n \right)} \approx 0
    \end{gather*}
    \subsection{Poisson Distribution}
    A Poisson process has a Poisson distribution with rate \(\eta\), so that
    \(X\left( t \right) \sim \operatorname{Poisson}{\left( \eta t \right)}\).
    \begin{equation*}
        \Pr{\left( X\left( t \right) = n \right)} = p_n\left( t \right) = \frac{e^{-\eta t} \left( \eta t \right)^n}{n!}
    \end{equation*}
    for \(n \geq 0\), where \(\eta t\) is the expected number of events.
    
    The number of events occurring between
    \(t_1\) and \(t_2\) is given by \(N\left( t_1,\: t_2 \right) \sim \operatorname{Poisson}{\left( \eta \left( t_2 - t_1 \right) \right)}\).
    \begin{flalign*}
        \Pr\left( N\left( t_1,\: t_2 \right) = n \right) = \\
        \frac{e^{-\eta \left( t_2 - t_1 \right)} \left( \eta \left( t_2 - t_1 \right) \right)^n}{n!}
    \end{flalign*}
    \subsection{Exponential Distribution}
    Let \(T\) be the time between events of a Poisson process so that \(T\) has an exponential distribution
    \begin{equation*}
        T \sim \operatorname{Exp}{\left( \eta \right)}.
    \end{equation*}
    The probability density function of \(T\) is given by
    \begin{equation*}
        f\left( t \right) = \eta e^{-\eta t}
    \end{equation*}
    for \(t > 0\).
    \subsection{Properties of Poisson Processes}
    \begin{enumerate}
        \setlength\itemsep{-0.2em}
        \item As the time between Poisson processes has an exponential distribution, the Poisson process inherits the
              memoryless property,
              \begin{flalign*}
                  \Pr{\left( T > x + y \,\vert\, T > x \right)} = \\
                  \Pr{\left( T > y \right)}.
              \end{flalign*}
        \item Non-overlapping time intervals of a Poisson process are independent. For \(a < b\) and \(c < d\) where \(b \leq c\),
              \begin{flalign*}
                  \Pr{\left( N\left( a,\: b \right) = m \,\vert\, N\left( c,\: d \right) = n \right)} = \\
                  \Pr{\left( N\left( a,\: b \right) = m \right)}
              \end{flalign*}
              \item\label{poisson_property_1_event} If exactly 1 event occurs on the interval \(\ointerval{0}{a}\), the distribution of when that event occurs is
              uniform. Let \(X\) be the time \(x < a\) when the first event occurs,
              \begin{flalign*}
                  X \,\vert\, \left( N\left( 0,\: a \right) = 1 \right) \sim \\
                  \operatorname{Uniform}{\left( 0,\: a \right)}
              \end{flalign*}
              \item\label{poisson_property_n_events} If exactly \(n\) events occur on the interval \(\ointerval{0}{a}\), then the distribution of the number of events
              that occur in \(\ointerval{0}{s}\) is binomial, for \(s < t\). Let \(X\) be the number of events that occur in \(\ointerval{0}{s}\) for \(s < t\),
              \begin{flalign*}
                  X \,\vert\, \left( N\left( 0,\: a \right) = n \right) \sim \\
                  \operatorname{Binomial}{\left( n,\: \frac{s}{t} \right)}
              \end{flalign*}
    \end{enumerate}
\end{multicols}
\end{document}