%!TEX TS-program = xelatex
%!TEX options = -aux-directory=Debug -shell-escape -file-line-error -interaction=nonstopmode -halt-on-error -synctex=1 "%DOC%"
\documentclass{article}
\input{LaTeX-Submodule/template.tex}

% Additional packages & macros
\usepackage{changepage} % Modify page width
\usepackage{multicol} % Use multiple columns
\usepackage[explicit]{titlesec} % Modify section heading styles

\titleformat{\section}{\normalfont\bfseries}{}{0em}{#1}
\titleformat{\subsection}{\normalfont\small\bfseries}{}{0em}{#1}

%% A4 page
\geometry{
	a4paper,
	margin = 10mm
}

%% Hide horizontal rule 
\renewcommand{\headrulewidth}{0pt}

%% Hide page numbers
\pagenumbering{gobble}

%% Multi-columns setup
\setlength\columnsep{4pt}

%% Paragraph setup
\setlength\parindent{0pt}
\setlength\parskip{0pt}

%% Customise section heading styles
% \titleformat*\section{\raggedright\bfseries}

\begin{document}
% Modify spacing
\titlespacing*\section{0pt}{1ex}{1ex}
\titlespacing*\subsection{0pt}{1ex}{1ex}
%
\setlength\abovecaptionskip{-5pt}
\setlength\textfloatsep{0pt}
%
\setlength\abovedisplayskip{1pt}
\setlength\belowdisplayskip{1pt}

\begin{multicols}{3}
    \section{Events and Probability}
    \subsection{Event}
    A set of outcomes from a random experiment.
    \subsection{Sample Space}
    Set of all possible outcomes \(\Omega\).
    \subsection{Intersection}
    Outcomes occur in both \(A\) and \(B\)
    \begin{equation*}
        A \cap B \quad\quad \text{or} \quad\quad AB
    \end{equation*}
    \subsection{Disjoint}
    Two events cannot occur simultaneously or have no common outcomes
    \begin{equation*}
        AB = \varnothing
    \end{equation*}
    These events are dependent.
    \subsection{Union}
    Set of outcomes in either \(A\) or \(B\)
    \begin{equation*}
        A \cup B
    \end{equation*}
    \subsection{Complement}
    Set of all outcomes not in \(A\), but in \(\Omega\) --- \(\overline{A} = \Omega \backslash A\).
    \begin{align*}
        A\overline{A}       & = \varnothing \\
        A \cup \overline{A} & = \Omega
    \end{align*}
    \subsection{Subset}
    \(A\) is a (non-strict) subset of \(B\) if all elements in \(A\) are also in \(B\) --- \(A \subset B\).
    \begin{equation*}
        AB = A \quad\quad \text{and} \quad\quad A \cup B = B
    \end{equation*}
    \begin{equation*}
        \forall A:A\subset \Omega \land \varnothing \subset A
    \end{equation*}
    \subsection{Identities}
    \begin{align*}
        A \left( BC \right)            & = \left( AB \right) C                             \\
        A \cup \left( B \cup C \right) & = \left( A \cup B \right) \cup C                  \\
        A \left(B \cup C\right)        & = AB \cup AC                                      \\
        A \cup BC                      & = \left( A \cup B \right) \left( A \cup C \right)
    \end{align*}
    \section{Probability}
    Measure of the likeliness of an event occurring
    \begin{equation*}
        \Pr{\left( A \right)} \quad\quad \text{or} \quad\quad \mathrm{P}\left( A \right)
    \end{equation*}
    \begin{equation*}
        0 \leq \Pr{\left( E \right)} \leq 1
    \end{equation*}
    where a probability of 0 never happens, and 1 always happens.
    \begin{align*}
        \Pr{\left( \Omega \right)}       & = 1                         \\
        \Pr{\left( \overline{E} \right)} & = 1 - \Pr{\left( E \right)}
    \end{align*}
    \subsection{Multiplication Rule}
    For independent events \(A\) and \(B\)
    \begin{equation*}
        \Pr{\left( AB \right)} = \Pr{\left( A \right)} \Pr{\left( B \right)}.
    \end{equation*}
    For dependent events \(A\) and \(B\)
    \begin{equation*}
        \Pr{\left( AB \right)} = \Pr{\left( A \,\vert\, B \right)} \Pr{\left( B \right)}
    \end{equation*}
    \subsection{Addition Rule}
    For independent \(A\) and \(B\)
    \begin{equation*}
        \Pr{\left( A \cup B \right)} = \Pr{\left( A \right)} + \Pr{\left( B \right)} - \Pr{\left( AB \right)}.
    \end{equation*}
    If \(AB = \varnothing\), then \(\Pr{\left( AB \right)} = 0\), so that \(\Pr{\left( A \cup B \right)} = \Pr{\left( A \right)} + \Pr{\left( B \right)}\).
    \subsection{De Morgan's Laws}
    \begin{align*}
        \overline{A \cup B} & = \overline{A} \ \overline{B}     \\
        \overline{AB}       & = \overline{A} \cup \overline{B}.
    \end{align*}
    \begin{align*}
        \Pr{\left( A \cup B \right)} & = 1 - \Pr{\left( \overline{A} \ \overline{B} \right)}    \\
        \Pr{\left( AB \right)}       & = 1 - \Pr{\left( \overline{A} \cup \overline{B} \right)}
    \end{align*}
    % ! Shorten this section about circuits
    % \subsection{Circuits}
    % A signal can pass through a circuit if there is a functional path from start to finish.

    % We can define a circuit where each component \(i\) functions with probability \(p\),
    % and is independent of other components.

    % Then \(W_i\) to be the event in which the associated component \(i\) functions, we can
    % determine the event \(S\) in which the system functions,
    % and probability \(\Pr{\left( S \right)}\) that the system functions.

    % As the probability that any component functions is \(p\), in other words
    % \begin{equation*}
    %     \Pr{\left( W_i \right)} = p,
    % \end{equation*}
    % \(\Pr{\left( S \right)}\) will be a function of \(p\) defined \(f:\left[ 0,\; 1 \right] \to \left[ 0,\; 1 \right]\).
    \section{Independence}
    \subsection{Conditional probability}
    The probability of event \(A\) given \(B\) has already occurred
    \begin{equation*}
        \Pr{\left( A \,\vert\, B \right)} = \frac{\Pr{\left( A B \right)}}{\Pr{\left( B \right)}}.
    \end{equation*}
    \(A\) and \(B\) are independent events if
    \begin{align*}
        \Pr{\left( A \,\vert\, B \right)} & = \Pr{\left( A \right)} \\
        \Pr{\left( B \,\vert\, A \right)} & = \Pr{\left( B \right)}
    \end{align*}
    the following statements are also true
    \begin{align*}
        \Pr{\left( A \,\vert\, \overline{B} \right)}            & = \Pr{\left( A \right)}            \\
        \Pr{\left( \overline{A} \,\vert\, B \right)}            & = \Pr{\left( \overline{A} \right)} \\
        \Pr{\left( \overline{A} \,\vert\, \overline{B} \right)} & = \Pr{\left( \overline{A} \right)}
    \end{align*}
    \subsection{Probability Rules with Conditional}
    All probability rules hold when conditioning on another event \(C\).
    \begin{align*}
        \Pr{\left( \overline{A} \,\vert\, C \right)} & = 1 - \Pr{\left( A \,\vert\, C \right)}                                 \\
        \Pr{\left( A \cup B \,\vert\, C \right)}     & = \begin{aligned}[t]
                                                             \Pr{\left( A \,\vert\, C \right)} + \Pr{\left( B \,\vert\, C \right)} \\
                                                             - \Pr{\left( AB \,\vert\, C \right)}
                                                         \end{aligned} \\
        \Pr{\left( A B \,\vert\, C \right)}          & = \Pr{\left( A \,\vert\, BC \right)} \Pr{\left( B \,\vert\, C \right)}
    \end{align*}
    \subsection{Conditional Independence}
    Given \(\Pr{\left( A \,\vert\, B \right)} \neq \Pr{\left( A \right)}\)
    \(A\) and \(B\) are conditionally dependent given \(C\) if
    \begin{equation*}
        \Pr{\left( A \,\vert\, BC \right)} = \Pr{\left( A \,\vert\, C \right)}.
    \end{equation*}
    Futhermore
    \begin{equation*}
        \Pr{\left( AB \,\vert\, C \right)} = \Pr{\left( A \,\vert\, C \right)} \Pr{\left( B \,\vert\, C \right)}.
    \end{equation*}
    Conversely
    \begin{align*}
        \Pr{\left( A \,\vert\, B \right)}  & = \Pr{\left( A \right)}                                                \\
        \Pr{\left( A \,\vert\, BC \right)} & \neq \Pr{\left( A \,\vert\, C \right)}                                 \\
        \Pr{\left( AB \,\vert\, C \right)} & = \Pr{\left( A \,\vert\, BC \right)} \Pr{\left( B \,\vert\, C \right)}
    \end{align*}
    Pairwise independence does not imply mutual independence
    \begin{gather*}
        \begin{cases}
            \Pr{\left( A B \right)} = \Pr{\left( A \right)} \Pr{\left( B \right)} \\
            \Pr{\left( A C \right)} = \Pr{\left( A \right)} \Pr{\left( C \right)} \\
            \Pr{\left( B C \right)} = \Pr{\left( B \right)} \Pr{\left( C \right)}
        \end{cases} \not\Rightarrow                                                                          \\
        \Pr{\left( A B C \right)} = \Pr{\left( A \right)} \Pr{\left( B \right)} \Pr{\left( C \right)}.
    \end{gather*}
    Independence should not be assumed unless explicitly stated.
    \subsection{Disjoint Events}
    Given \(AB = \varnothing\)
    \begin{align*}
        \Pr{\left( AB \right)}            & = 0 \implies \Pr{\left( \varnothing \right)} = 0 \\
        \Pr{\left( A \,\vert\, B \right)} & = 0
    \end{align*}
    \subsection{Subsets}
    If \(A \subset B\) then \(\Pr{\left( A \right)} \leq \Pr{\left( B \right)}\).
    \begin{align*}
        \Pr{\left( B \,\vert\, A \right)} & = 1                                                   \\
        \Pr{\left( A \,\vert\, B \right)} & = \frac{\Pr{\left( A \right)}}{\Pr{\left( B \right)}}
    \end{align*}
    These events are also highly dependent.
    \section{Total Probability}
    \subsection{Marginal probability}
    Marginal probability is the probability of an event \linebreak irrespective of the outcome of another variable.
    \begin{theorem}[Total probability for complements]
        By writing the event \(A\) as \(AB \cup A\overline{B}\), and noting that \(AB\) and \(A\overline{B}\) are disjoint,
        the marginal probability of \(A\) is given by
        \begin{equation*}
            \Pr{\left( A \right)} = \Pr{\left( AB \right)} + \Pr{\left( A\overline{B} \right)}.
        \end{equation*}
        By applying the multiplication rule to each joint probability:
        \begin{equation*}
            \Pr{\left( A \right)} = \Pr{\left( A \,\vert\, B \right)}\Pr{\left( B \right)} + \Pr{\left( A \,\vert\, \overline{B} \right)}\Pr{\left( \overline{B} \right)}
        \end{equation*}
    \end{theorem}
    \begin{theorem}[Law of total probability]
        The previous theorem partitioned \(\Omega\) into disjoint events \(B\) and \(\overline{B}\).

        By partitioning \(\Omega\) into a collection of disjoint events \(B_1,\; B_2,\; \dots,\; B_n\),
        such that \(\bigcup_{i=1}^n B_i = \Omega\), we have
        \begin{equation*}
            \Pr{\left( A \right)} = \sum_{i = 1}^n \Pr{\left( A \,\vert\, B_i \right)}\Pr{\left( B_i \right)}
        \end{equation*}
    \end{theorem}
    \subsection{Bayes' Theorem}
    \begin{equation*}
        \Pr{\left( A \,\vert\, B \right)} = \frac{\Pr{\left( B \,\vert\, A \right)}\Pr{\left( A \right)}}{\Pr{\left( B \right)}}
    \end{equation*}
    \section{Combinatorics}
    \subsection{Number of outcomes}
    Let \(\abs{A}\) denote the number of outcomes in an event \(A\).
    \begin{theorem}[Addition principle]
        Given a sample space \(S\) with \(k\) disjoint events \({\left\{ S_1,\:\ldots,\:S_k \right\}}\),
        where the \(i\)th event has \(n_i\) possible outcomes,
        the number of possible samples from any event is given by
        \begin{equation*}
            \abs{\bigcup_{i = 0}^{k} S_i} = \sum_{i = 1}^k n_i
        \end{equation*}
    \end{theorem}
    \begin{theorem}[Multiplication principle]
        Given a sample space \(S\) with \(k\) events \({\left\{ S_1,\:\ldots,\:S_k \right\}}\),
        where the \(i\)th event has \(n_i\) possible outcomes,
        the number of possible samples from every event is given by
        \begin{equation*}
            \abs{\bigcap_{i=0}^{k} S_i} = \prod_{i = 1}^k n_i
        \end{equation*}
    \end{theorem}
    \begin{theorem}[Counting probability]
        Given a sample space \(S\) with equally likely outcomes, the probability
        of an event \(S_i \subset S\) is given by
        \begin{equation*}
            \Pr{\left( S_i \right)} = \frac{\abs{S_i}}{\abs{S}}
        \end{equation*}
    \end{theorem}
    \subsection{Ordered Sampling with Replacement}
    When ordering is important and repetition is allowed,
    the total number of ways to choose \(k\) objects from a set with \(n\) elements is
    \begin{equation*}
        n^k
    \end{equation*}
    \subsection{Ordered Sampling without Replacement}
    When ordering is important and repetition is not allowed,
    the total number of ways to arrange \(k\) objects from a set of \(n\) elements is
    known as a \(k\)-permutation of \(n\)-elements denoted \(\prescript{n}{}{P}_k\)
    \begin{align*}
        \prescript{n}{}{P}_k & = n \times \left( n - 1 \right) \times \cdots \times \left( n - k + 1 \right) \\
                             & = \frac{n!}{\left( n - k \right)!}
    \end{align*}
    for \(0 \leq k \leq n\).
    \subsection{Permutation of \(n\) elements}
    An \(n\)-permutation of \(n\) elements is the permutation of those elements.
    In this case, \(k = n\), so that
    \begin{align*}
        \prescript{n}{}{P}_n & = n \times \left( n - 1 \right) \times \cdots \times \left( n - n + 1 \right) \\
                             & = n!
    \end{align*}
    \subsection{Unordered Sampling without Replacement}
    When ordering is not important and repetition is not allowed,
    the total number of ways to choose \(k\) objects from a set of \(n\) elements is
    known as a \(k\)-combination of \(n\)-elements denoted \(\prescript{n}{}{C}_k\) or \(\binom{n}{k}\)
    \begin{align*}
        \prescript{n}{}{C}_k & = \frac{\prescript{n}{}{P}_k}{k!}     \\
                             & = \frac{n!}{k! \left( n - k \right)!}
    \end{align*}
    for \(0 \leq k \leq n\). We divide by \(k!\) because any \(k\)-element subset of \(n\)-elements % chktex 40
    can be ordered in \(k!\) ways. % chktex 40
    \subsection{Unordered Sampling with Replacement}
    When ordering is not important and repetition is allowed,
    the total number of ways to choose \(k\) objects from a set with \(n\) elements is
    \begin{equation*}
        \binom{n + k - 1}{k}
    \end{equation*}
    \section{Random Variables and Distributions}
    \section{Random Variables}
    \subsection{Random variable}
    A random variable \(X\) is a measurable variable whose value holds some uncertainty.
    An event is when a random variable assumes a certain value or range of values.
    \subsection{Discrete random variables}
    A discrete random variable takes discrete values.
    \subsection{Continuous random variables}
    A continuous random variable can take any real value.
    \subsection{Probability Distributions}
    \subsection{Probability distribution}
    The probability distribution of a random variable \(X\) is a function that links all outcomes \(x \in \Omega\)
    to the probability that they will occur \(\Pr{\left( X = x \right)}\).
    \subsection{Probability mass function}
    The probability distribution of a discrete random variable \(X\) is described by a Probability
    Mass Function (PMF) \(p_x\).
    \begin{equation*}
        \Pr{\left( X = x \right)} = p_x
    \end{equation*}
    \(p_x\) is a valid PMF provided,
    \begin{align*}
        \forall x \in \Omega : \Pr{\left( X = x \right)} & \geq 0 &  & \text{and} & \sum_{x \in \Omega} \Pr{\left( X = x \right)} & = 1.
    \end{align*}
    \subsection{Probability density function}
    The probability distribution of a continuous random variable \(X\) is described by a Probability
    Density Function (PDF) \(f\left( x \right)\).

    The probability that \(X\) is exactly equal to a
    specific value is always 0. Therefore we compute probabilities over intervals:
    \begin{equation*}
        \Pr{\left( x_1 \leq X \leq x_2 \right)} = \int_{x_1}^{x_2} f\left( x \right) \odif{x}
    \end{equation*}
    \(f\left( x \right)\) is a valid PDF provided,
    \begin{align*}
        \forall x \in \Omega : f\left( x \right) & \geq 0 &  & \text{and} & \int_{\Omega} f(x) \odif{x} & = 1.
    \end{align*}
    \subsection{Cumulative distribution function}
    The Cumulative Distribution Function (CDF) computes the probability that the random variable is
    less than or equal to a particular realisation \(x\). For \(U = \left\{ k \in \Omega : k \leq x \right\}\)
    \begin{equation*}
        F\left( x \right) = \Pr{\left( X \leq x \right)} = \begin{cases}
            \displaystyle \sum_{u \in U} p_u                  & \text{for discrete random variables}    \\[0.4cm]
            \displaystyle \int_{U} f\left( u \right) \odif{u} & \text{for continuous random variables}.
        \end{cases}
    \end{equation*}
    \(F\left( x \right)\) is a valid CDF if:
    \begin{enumerate}
        \item \(F\) is monotonically increasing and continuous
        \item \(\lim_{x \to -\infty} F\left( x \right) = 0\)
        \item \(\lim_{x \to \infty} F\left( x \right) = 1\)
    \end{enumerate}
    We can recover the PDF given the CDF, by using the Fundamental Theorem of Calculus.
    \begin{equation*}
        \odv{F\left( x \right)}{x} = \odv{}{x} \int_{-\infty}^x f\left( u \right) \odif{u} = f\left( x \right)
    \end{equation*}
    \subsection{Complementary CDF}
    For a continuous random variable \(X\) the complement function,
    \begin{equation*}
        \Pr{\left( X > x \right)} = 1 - \Pr{\left( X \leq x \right)} = 1 - F\left( x \right)
    \end{equation*}
    is called the complementary CDF, or the survival function.
    \subsection{Quantiles}
    \subsection{\(p\)-Quantile}
    For a continuous random variable, the \(p\)-quantile, \(x\), is defined such that
    \begin{equation*}
        F\left( x \right) = \int_{-\infty}^x f\left( u \right) \odif{u} = p.
    \end{equation*}
    \subsection{Median}
    The median, \(m\), is a special \(p\)-quartile defined as the value such that
    \begin{equation*}
        \int_{-\infty}^m f\left( u \right) \odif{u} = \int^{\infty}_m f\left( u \right) \odif{u} = \frac{1}{2}.
    \end{equation*}
    \subsection{Lower and upper quartile}
    Likewise the lower quartile and upper quartiles are two values \(q_1\) and \(q_2\) such that
    \begin{equation*}
        \int_{-\infty}^{q_1} f\left( u \right) \odif{u} = \frac{1}{4}
    \end{equation*}
    and
    \begin{equation*}
        \int_{-\infty}^{q_2} f\left( u \right) \odif{u} = \frac{3}{4}.
    \end{equation*}
    \subsection{Quantile function}
    The quantile function is the inverse of the CDF
    and can be used to find the \(x\) that a certain \(p\) provides. I.e.,
    \begin{equation*}
        x = F^{-1}\left( p \right) = Q\left( p \right)
    \end{equation*}
    \subsection{Summary Statistics}
    \subsection{Expectation}
    The expectation \(\E{\left( X \right)}\), or \(\operatorname{\mathbb{E}}{\left( X \right)}\)
    of a random variable \(X\) is its expected value given an
    infinite number of observations.

    The expectation is also known as the mean of the \(X\), denoted \(\mu\).
    \begin{equation*}
        \E{\left( X \right)} =
        \begin{cases}
            \displaystyle \sum_{x \in \Omega} x p_x                & \text{for discrete variables}   \\[0.4cm]
            \displaystyle \int_\Omega x f\left( x \right) \odif{x} & \text{for continuous variables}
        \end{cases}
    \end{equation*}
    \begin{theorem}
        Using integration by parts, it can be proved that
        \begin{equation*}
            \E{\left(X\right)} = -\int_{-\infty}^0 F\left( x \right) \odif{x} + \int_0^\infty \left(1 - F\left( x \right)\right) \odif{x}
        \end{equation*}
    \end{theorem}
    \subsection{Variance}
    The variance \(\Var{\left( X \right)}\), or \(\operatorname{\mathbb{V}}{\left( X \right)}\) of a random variable \(X\) is a measure of spread
    of the distribution (defined as the average squared distance of each value from the mean).
    Variance is also denoted as \(\sigma^2\).
    \begin{align*}
        \Var{\left( X \right)} & =
        \begin{cases}
            \displaystyle \sum_{x \in \Omega} \left( x - \mu \right)^2 p_x                  & \text{for discrete variables}   \\[0.4cm]
            \displaystyle \int_{\Omega} \left( x - \mu \right)^2 f\left( x \right) \odif{x} & \text{for continuous variables}
        \end{cases} \\
                               & = \E{\left( X^2 \right)} - \E{\left( X \right)}^2
    \end{align*}
    \subsection{Standard deviation}
    The standard deviation is defined as
    \begin{equation*}
        \sigma = \sqrt{\Var{\left( X \right)}}
    \end{equation*}
    \subsubsection{Transformations}
    For a simple linear function of a random variable
    \begin{align*}
        \E{\left( aX \pm b \right)}   & = a\E{\left( X \right)} \pm b \\
        \Var{\left( aX \pm b \right)} & = a^2\Var{\left( X \right)}
    \end{align*}
    \section{Special Discrete Distributions}
    \subsection{Discrete Uniform Distribution}
    A discrete uniform distribution describes the probability distribution of a single trial
    in a set of equally likely elements.

    A discrete random variable \(X\) with a discrete uniform distribution is denoted
    \begin{equation*}
        X \sim \operatorname{Uniform}{\left( a,\: b \right)}
    \end{equation*}
    with
    \begin{align*}
        \Pr{\left( X = x \right)}    & = \frac{1}{b - a + 1}         \\
        \Pr{\left( X \leq x \right)} & = \frac{x - a + 1}{b - a + 1}
    \end{align*}
    for outcomes \(x \in \left\{ a,\: a + 1,\: \dots,\: b - 1,\: b \right\}\).
    We can also summarise the following:
    \begin{align*}
        \E{\left( X \right)}   & = \frac{a + b}{2}                           \\
        \Var{\left( X \right)} & = \frac{\left( b - a + 1 \right)^2 - 1}{12}
    \end{align*}
    \subsection{Bernoulli Distribution}
    A Bernoulli (or binary) distribution describes the probability distribution of a Boolean-valued
    outcome, i.e., success (1) or failure (0).

    A discrete random variable \(X\) with a Bernoulli distribution is denoted
    \begin{equation*}
        X \sim \operatorname{Bernoulli}{\left( p \right)}
    \end{equation*}
    with
    \begin{align*}
        \Pr{\left( X = x \right)}    & = \begin{cases}
                                             1 - p & x = 0 \\
                                             p     & x = 1
                                         \end{cases}                    \\
                                     & = p^x \left( 1 - p \right)^{1 - x} \\
        \Pr{\left( X \leq x \right)} & = \begin{cases}
                                             0     & x < 0        \\
                                             1 - p & 0 \leq x < 1 \\
                                             1     & k \geq 1
                                         \end{cases}
    \end{align*}
    for a probability \(p \in \interval{0}{1}\) and outcomes \(x \in \left\{ 0,\: 1 \right\}\).
    We can also summarise the following:
    \begin{align*}
        \E{\left( X \right)}   & = p                      \\
        \Var{\left( X \right)} & = p \left( 1 - p \right)
    \end{align*}
    where \(\left( 1 - p \right)\) is sometimes denoted as \(q\).
    \subsection{Binomial Distribution}
    A binomial distribution describes the probability distribution of the number of successes
    for \(n\) independent trials with the same probability of success \(p\).

    A discrete random variable \(X\) with a binomial distribution is denoted
    \begin{equation*}
        X \sim \operatorname{B}{\left( n,\: p \right)}
    \end{equation*}
    with
    \begin{align*}
        \Pr{\left( X = x \right)}    & = \dbinom{n}{x} p^x \left( 1 - p \right)^{n - x}                \\
        \Pr{\left( X \leq x \right)} & = \sum_{u = 0}^x \dbinom{n}{u} p^u \left( 1 - p \right)^{n - u}
    \end{align*}
    for number of successes \(x \in \left\{ 0,\: 1,\: \dots,\: n \right\}\).

    Here each individual trial is a Bernoulli trial, so that \(X\) can be written as the sum of
    \(n\) \textit{independent and identically distributed} (iid) Bernoulli random variables, \(Y_1,\: Y_2,\: \dots,\: Y_n\).
    \begin{align*}
        X & = Y_1 + Y_2 + \cdots + Y_n \\
        Y_i & \overset{\mathrm{iid}}{\sim} \operatorname{Bernoulli}{\left( p \right)} : \forall i \in \left\{ 1,\: 2,\: \dots,\: n \right\}.
    \end{align*}
    We can then summarise the following:
    \begin{align*}
        \E{\left( X \right)}   & = np                      \\
        \Var{\left( X \right)} & = np \left( 1 - p \right)
    \end{align*}
    \subsection{Geometric Distribution}
    A geometric distribution describes the probability distribution of the number of trials up to and including
    the first success, where each trial is independent and has the same probability of success \(p\).

    A discrete random variable \(N\) with a geometric distribution is denoted
    \begin{equation*}
        N \sim \operatorname{Geom}{\left( p \right)}
    \end{equation*}
    with
    \begin{align*}
        \Pr{\left( N = n \right)}    & = \left( 1 - p \right)^{n - 1} p \\
        \Pr{\left( N \leq n \right)} & = 1 - \left( 1 - p \right)^n
    \end{align*}
    for number of trials \(n \geq 1\).

    We can also summarise the following:
    \begin{align*}
        \E{\left( N \right)}   & = \frac{1}{p}       \\
        \Var{\left( N \right)} & = \frac{1 - p}{p^2}
    \end{align*}
    \subsubsection{Alternate Geometric Definition}
    We can alternatively consider the number of failures until a success, \(Y\):
    \begin{equation*}
        Y = N - 1
    \end{equation*}
    Therefore the PMF and CDF for \(Y\) are:
    \begin{align*}
        \Pr{\left( Y = y \right)}    & = \left( 1 - p \right)^y p         \\
        \Pr{\left( Y \leq y \right)} & = 1 - \left( 1 - p \right)^{y + 1}
    \end{align*}
    for number of failures \(y \geq 0\). This gives the following summary statistics using
    transformation rules:
    \begin{align*}
        \E{\left( Y \right)}   & = \frac{1 - p}{p}   \\
        \Var{\left( Y \right)} & = \frac{1 - p}{p^2}
    \end{align*}
    \subsection{Negative Binomial Distribution}
    A negative binomial distribution describes the probability distribution of the number of trials until \(k \geq 1\)
    successes, where each trial is independent and has the same probability of success \(p\).

    A discrete random variable \(N\) with a negative binomial distribution is denoted
    \begin{equation*}
        N \sim \operatorname{NB}{\left( k,\: p \right)}
    \end{equation*}
    with
    \begin{align*}
        \Pr{\left( N = n \right)}    & = \dbinom{n - 1}{k - 1} \left( 1 - p \right)^{n - k} p^k                \\
        \Pr{\left( N \leq n \right)} & = \sum_{u = k}^n \dbinom{u - 1}{k - 1} \left( 1 - p \right)^{u - k} p^k
    \end{align*}
    for number of trials \(n \geq k\).
    Here each individual trial is a Geometric trial, so that \(N\) can be written as the sum of
    \(k\) \textit{independent and identically distributed} (iid) Geometric random variables, \(Y_1,\: Y_2,\: \dots,\: Y_k\).
    \begin{align*}
        N & = Y_1 + Y_2 + \cdots + Y_k, & Y_i & \overset{\mathrm{iid}}{\sim} \operatorname{Geom}{\left( p \right)} : \forall i \in \left\{ 1,\: 2,\: \dots,\: k \right\}.
    \end{align*}
    We can then summarise the following:
    \begin{align*}
        \E{\left( N \right)}   & = \frac{k}{p}                        \\
        \Var{\left( N \right)} & = \frac{k \left( 1 - p \right)}{p^2}
    \end{align*}
    \subsubsection{Alternate Negative Binomial Definition}
    We can alternatively consider the number of failures \(Y\) until \(k\) successes:
    \begin{equation*}
        Y = N - k
    \end{equation*}
    The PMF and CDF for \(Y\) are given by:
    \begin{align*}
        \Pr{\left( Y = y \right)}    & = \dbinom{y + k - 1}{k - 1} \left( 1 - p \right)^y p^k                \\
        \Pr{\left( Y \leq y \right)} & = \sum_{u = 0}^y \dbinom{u + k - 1}{k - 1} \left( 1 - p \right)^u p^k
    \end{align*}
    for number of failures \(y \geq 0\). This gives the following summary statistics using
    transformation rules:
    \begin{align*}
        \E{\left( Y \right)}   & = \frac{k \left( 1 - p \right)}{p}   \\
        \Var{\left( Y \right)} & = \frac{k \left( 1 - p \right)}{p^2}
    \end{align*}
    \subsection{Poisson Distribution}
    A Poisson distribution describes the probability distribution of the number of events \(N\) which occur over a fixed interval of time \(\lambda\).

    A discrete random variable \(N\) with a Poisson distribution is denoted
    \begin{equation*}
        N \sim \operatorname{Pois}{\left( \lambda \right)}
    \end{equation*}
    with
    \begin{align*}
        \Pr{\left( N = n \right)}    & = \frac{\lambda^n e^{-\lambda}}{n!}                \\
        \Pr{\left( N \leq n \right)} & = e^{-\lambda} \sum_{u = 0}^n \frac{\lambda^u}{u!}
    \end{align*}
    for number of events \(n \geq 0\).
    We can also summarise the following:
    \begin{align*}
        \E{\left( N \right)}   & = \lambda \\
        \Var{\left( N \right)} & = \lambda
    \end{align*}
    \subsection{Modelling Count Data}
    If we want to utilise these distributions to model data, we can use the following observations:
    \begin{itemize}
        \item Poisson (mean = variance)
        \item Binomial (underdispersed, mean > variance)
        \item Geometric/Negative Binomial (overdispersed, mean < variance)
    \end{itemize}
    \section{Special Continuous Distributions}
    \subsection{Continuous Uniform Distribution}
    A continuous uniform distribution describes the probability distribution of an outcome within some
    interval, where the probability of an outcome in one interval is the same as all other intervals of the same length.

    A continuous random variable \(X\) with a continuous uniform distribution is denoted
    \begin{equation*}
        X \sim \operatorname{U}{\left( a,\: b \right)}
    \end{equation*}
    with
    \begin{align*}
        f\left( x \right) & = \frac{1}{b - a}     \\
        F\left( x \right) & = \frac{x - a}{b - a}
    \end{align*}
    for outcomes \(a < x < b\).
    We can also summarise the following:
    \begin{align*}
        \E{\left( X \right)}   & = \frac{a + b}{2}                   \\
        \Var{\left( X \right)} & = \frac{\left( b - a \right)^2}{12} \\
        m                      & = \frac{a + b}{2}
    \end{align*}
    \subsection{Exponential Distribution}
    An exponential distribution describes the probability distribution of the time between events with rate \(\eta\).

    A continuous random variable \(T\) with an exponential distribution is denoted
    \begin{equation*}
        T \sim \operatorname{Exp}{\left( \eta \right)}
    \end{equation*}
    with
    \begin{align*}
        f\left( t \right) & = \eta e^{-\eta t} \\
        F\left( t \right) & = 1 - e^{-\eta t}
    \end{align*}
    for time \(t > 0\).
    We can also summarise the following:
    \begin{align*}
        \E{\left( X \right)}   & = \frac{1}{\eta}                     \\
        \Var{\left( X \right)} & = \frac{1}{12}                       \\
        m                      & = \frac{\ln{\left( 2 \right)}}{\eta}
    \end{align*}
    \subsection{Memoryless Property}
    In an exponential distribution with \(T \sim \operatorname{Exp}{\left( \eta \right)}\),
    the distribution of the waiting time \(t + s\) until a certain event does not depend on
    how much time \(t\) has already passed.
    \begin{equation*}
        \Pr{\left( T > s + t \,\vert\, T > t \right)} = \Pr{\left( T > s \right)}.
    \end{equation*}
    The same property also applies in an Geometric distribution with \(N \sim \operatorname{Geom}{\left( p \right)}\).
    \subsection{Normal Distribution}
    The normal distribution is used to represent many random situations, in particular, measurements and their errors.
    This distribution arises in many statistical problems and can be used to \linebreak approximate other distributions
    under certain conditions.

    A continuous random variable \(X\) with a normal distribution is denoted
    \begin{equation*}
        X \sim \operatorname{N}{\left( \mu,\: \sigma^2 \right)}
    \end{equation*}
    with
    \begin{align*}
        f\left( t \right) & = \frac{1}{\sqrt{2 \pi \sigma^2}} e^{-\frac{\left( x - \mu \right)^2}{2 \sigma^2}}    \\
        F\left( t \right) & = \frac{1}{2} \left( 1 + \erf{\left( \frac{x - \mu}{\sigma \sqrt{2}} \right)} \right)
    \end{align*}
    for \(x \in \R\) where \(\erf{\left( z \right)} = \frac{2}{\sqrt{\pi}} \int_0^z e^{-t^2} \odif{t}\) is the error function.
    We can also summarise the following:
    \begin{align*}
        \E{\left( X \right)}   & = \mu      \\
        \Var{\left( X \right)} & = \sigma^2
    \end{align*}
    Given the complexity of the analytic expressions for the PDF and CDF of the normal distribution, we often
    use software to numerically determine probabilities associated with normal distributions.
    \subsection{Standard Normal Distribution}
    Given \(X \sim \operatorname{N}{\left( \mu,\: \sigma^2 \right)}\), consider the transformation
    \begin{equation*}
        Z = \frac{X - \mu}{\sigma}
    \end{equation*}
    so that \(Z \sim \operatorname{N}{\left( 0,\: 1 \right)}\). This distribution is called the standard normal distribution.
    This allows us to deal with the standard normal distribution regardless of \(\mu\) and \(\sigma\).
    \section{Central Limit Theorem}
    The central limit theorem states that the sum of independent and identically distributed random variables, when properly standardised,
    can be approximated by a normal distribution, as the number of elements increases.
    \subsection{Approximating the Average of Random Variables}
    Given a set of independent and identically distributed random variables \(X_1,\: \ldots,\: X_n\) from the distribution \(X\),
    if \(\E{\left( X \right)} = \mu\) and \(\Var{\left( X \right)} = \sigma^2\), then we can define \(\overline{X} = \frac{1}{n} \sum_{i = 1}^n X_i\)
    so that
    \begin{align*}
        \E{\left( \overline{X} \right)}   & = \mu                \\
        \Var{\left( \overline{X} \right)} & = \frac{\sigma^2}{n}
    \end{align*}
    By standardising \(\overline{X}\), we can define
    \begin{equation*}
        Z = \lim_{n \to \infty} \frac{\overline{X} - \mu}{\sigma / \sqrt{n}}
    \end{equation*}
    so that \(Z \to \operatorname{N}{\left( 0,\: 1 \right)}\) as \(n \to \infty\).
    \subsection{Approximating the Sum of Random Variables}
    Given a set of independent and identically distributed random variables \(X_1,\: \ldots,\: X_n\) from the distribution \(X\),
    if \(\E{\left( X \right)} = \mu\) and \(\Var{\left( X \right)} = \sigma^2\), then we can define \(\overline{Y} = \sum_{i = 1}^n X_i\)
    so that
    \begin{align*}
        \E{\left( Y \right)}   & = n \mu      \\
        \Var{\left( Y \right)} & = n \sigma^2
    \end{align*}
    Then for large \(n\)
    \begin{equation*}
        Y \sim \operatorname{N}{\left( n \mu,\: n \sigma^2 \right)}
    \end{equation*}
    \subsection{Approximating the Binomial Distribution}
    \subsubsection{Normal Distribution}
    Given a binomial distribution \(X \sim \operatorname{B}{\left( n,\: p \right)}\), we can write \(X\)
    as the sum of \(n\) independent and identically distributed Bernoulli random variables
    \(X_1,\: \ldots,\: X_n\), so that \(X_i \sim \operatorname{Bernoulli}{\left( p \right)}\).

    Thus by the central limit theorem, we can use a normal approximation for \(X\), provided that \(n\) is large.
    \begin{equation*}
        X \approx Y \sim \operatorname{N}{\left( np,\: np\left( 1 - p \right) \right)}
    \end{equation*}

    In general, this approximation is sufficient when \(np > 5\) and \(n\left( 1 - p \right) > 5\).
    \subsubsection{Poisson Distribution}
    When \(np < 5\) we can use the Poisson distribution to approximate \(X\) with the mean \(np\):
    \begin{equation*}
        X \approx Y \sim \operatorname{Pois}{\left( np \right)}.
    \end{equation*}
    When \(n\left( 1 - p \right) < 5\) we can consider the number of failures \(W = n - X\), so that,
    \begin{equation*}
        W \approx Y \sim \operatorname{Pois}{\left( n\left( 1 - p \right) \right)}.
    \end{equation*}
    \subsubsection{Continuity Correction}
    Given an approximation \(Y\) (either Normal or Poisson) for the binomial distribution \(X \sim \operatorname{B}{\left( n,\: p \right)}\) the equality
    \begin{equation*}
        \Pr{\left( X \leq x \right)} = \Pr{\left( X < x + 1 \right)}
    \end{equation*}
    must hold for any \(x\). Therefore by adding \(\frac{1}{2}\) we apply a continuity correction to the approximate probability:
    \begin{equation*}
        \Pr{\left( Y \leq x + \frac{1}{2} \right)}.
    \end{equation*}
    \subsection{Approximating a Poisson Distribution}
    Given a set of independent Poisson distributions \(X_1,\: \ldots ,\: X_n\) where \(X_i \sim \operatorname{Pois}{\left( \lambda \right)}\)
    so that \(\E{\left( X_i \right)} = \lambda\) and \(\Var{\left( X_i \right)} = \lambda\) for all \(i\).

    If we consider \(X = \sum_{i = 1}^n X_i\) then
    \begin{align*}
        \E{\left( X \right)}   & = n \lambda \\
        \Var{\left( X \right)} & = n \lambda
    \end{align*}
    so that by the central limit theorem, we can use the approximation
    \begin{equation*}
        X \approx Y \sim \operatorname{N}{\left( n\lambda,\: n\lambda \right)}.
    \end{equation*}
    In general, this approximation is sufficient when \(n \lambda > 10\), and when an accurate approximation is desired, \(n \lambda > 20\).
    \section{Bivariate Distributions}
    \subsection{Bivariate probability mass function}
    The distribution over the joint space of two discrete random variables \(X\) and \(Y\) is given by a bivariate probability mass function:
    \begin{equation*}
        \Pr{\left( X = x,\: Y = y \right)} = p_{x,\: y}
    \end{equation*}
    for all pairs of \(x \in \Omega_1\) and \(y \in \Omega_2\). This function must satisfy
    \begin{align*}
        \forall x \in \Omega_1 : \forall y \in \Omega_2 : \Pr{\left( X = x,\: Y = y \right)} \geq 0 &  & \text{and} &  &
        \sum_{y \in \Omega_2} \sum_{x \in \Omega_1} \Pr{\left( X = x,\: Y = y \right)} = 1.
    \end{align*}
    The joint probability mass function can be shown using a table:
    {\small
    \begin{equation*}
        \begin{matrix}[c|ccc] % chktex 44
                               & y_1                                    & \cdots & y_n                                    \\
            \hline % chktex 44
            x_1                & \Pr{\left( X = x_1,\: Y = y_1 \right)} & \cdots & \Pr{\left( X = x_1,\: Y = y_n \right)} \\
            \vdots             & \vdots                                 & \ddots & \vdots                                 \\
            x_n                & \Pr{\left( X = x_n,\: Y = y_1 \right)} & \cdots & \Pr{\left( X = x_n,\: Y = y_n \right)}
        \end{matrix}
    \end{equation*}
    }
    \subsection{Bivariate probability density function}
    The distribution over the joint space of two continuous random variables \(X\) and \(Y\) is  given by a
    bivariate probability density function \(f\left( x,\: y \right)\)
    over the intervals \(x \in \Omega_1\) and \(y \in \Omega_2\).
    \begin{equation*}
        \Pr{\left( x_1 \leq X \leq x_2,\: y_1 \leq Y \leq y_2 \right)} = \int_{x_1}^{x_2} \int_{y_1}^{y_2} f\left( x,\: y \right) \odif{y} \odif{x}
    \end{equation*}
    This function must satisfy
    \begin{align*}
        \forall x \in \Omega_1 : \forall y \in \Omega_2 : f\left( x,\: y \right) \geq 0 &  & \text{and} &  &
        \int_{x \in \Omega_1} \int_{y \in \Omega_2} f\left( x,\: y \right) \odif{y} \odif{x} = 1.
    \end{align*}
    \subsection{Marginal Probability}
    The marginal probability function can be obtained by calculating the probability function of each random variable.
    Once the function has been determined, we must specify the range of values that variable can take.
    \subsection{Marginal probability mass function}
    \begin{align*}
        \Pr{\left( X = x \right)} & = p_x = \sum_{y \in \Omega_2} \Pr{\left( X = x,\: Y = y \right)} \\
        \Pr{\left( Y = y \right)} & = p_y = \sum_{x \in \Omega_1} \Pr{\left( X = x,\: Y = y \right)}
    \end{align*}
    \subsection{Marginal probability density function}
    \begin{align*}
        \Pr{\left( X = x \right)} & = f\left( x \right) = \int_{y_1}^{y_2} f\left( x,\: y \right) \odif{y} \\
        \Pr{\left( Y = y \right)} & = f\left( y \right) = \int_{x_1}^{x_2} f\left( x,\: y \right) \odif{x}
    \end{align*}
    \subsection{Conditional Probability}
    Using the joint probability and marginal probability, we can determine the conditional probability function.
    Once the function has been determined, we must specify the range of values that variable can take.
    \subsection{Conditional probability mass function}
    \begin{equation*}
        \Pr{\left( X = x \,\vert\, Y = y \right)} = \frac{\Pr{\left( X = x,\: Y = y \right)}}{\Pr{\left( Y = y \right)}}.
    \end{equation*}
    It follows that
    \begin{equation*}
        \sum_{x \in \Omega_1} \Pr{\left( X = x \,\vert\, Y = y \right)} = 1
    \end{equation*}
    \subsection{Conditional probability density function}
    \begin{equation*}
        f\left( x \,\vert\, y \right) = \frac{f\left( x,\: y \right)}{f\left( y \right)}
    \end{equation*}
    It follows that
    \begin{equation*}
        \int_{x_1}^{x_2} f\left( x \,\vert\, y \right) \odif{x} = 1
    \end{equation*}
    \subsection{Independence}
    Two discrete random variables \(X\) and \(Y\) are independent if
    \begin{equation*}
        \Pr{\left( X = x \,\vert\, Y = y \right)} = \Pr{\left( X = x \right)}
    \end{equation*}
    for all pairs of \(x\) and \(y\). From this we can show that
    \begin{equation*}
        \Pr{\left( X = x ,\: Y = y \right)} = \Pr{\left( X = x \right)} \Pr{\left( Y = y \right)}
    \end{equation*}
    for all pairs of \(x\) and \(y\). If these random variables are not independent then,
    \begin{equation*}
        \Pr{\left( X = x ,\: Y = y \right)} = \Pr{\left( X = x \,\vert\, Y = y \right)} \Pr{\left( Y = y \right)}
    \end{equation*}
    To continuous random variables \(X\) and \(Y\) are independent if we can express \(f\left( x,\: y \right)\) as
    \begin{equation*}
        f\left( x,\: y \right) \propto g\left( x \right) h\left( y \right)
    \end{equation*}
    and if the joint range of \(X\) and \(Y\) do not depend on each other. This leads to
    \begin{equation*}
        f\left( x \,\vert\, y \right) = f\left( x \right).
    \end{equation*}
    \subsection{Law of Total Expectation}
    Given the conditional distribution of \(X \,\vert\, Y = y\), we can compute its expectation and variance.
    For discrete random variables, the conditional expectation is
    \begin{equation*}
        \E{\left( X \,\vert\, Y = y \right)} = \sum_{x\in\Omega_1} x \Pr{\left( X = x \,\vert\, Y = y \right)}
    \end{equation*}
    For continuous random variables, the conditional expectation is
    \begin{equation*}
        \E{\left( X \,\vert\, Y = y \right)} = \int_{x_1}^{x_2} x f\left( x \,\vert\, y \right) \odif{x}
    \end{equation*}
    The conditional variance is given by
    \begin{equation*}
        \Var{\left( X \,\vert\, Y = y \right)} = \E{\left( X^2 \,\vert\, Y = y \right)} - \E{\left( X \,\vert\, Y = y \right)}^2
    \end{equation*}
    When \(X\) and \(Y\) are independent,
    \begin{align*}
        \E{\left( X \,\vert\, Y = y \right)}   & = \E{\left( X \right)}   \\
        \Var{\left( X \,\vert\, Y = y \right)} & = \Var{\left( X \right)} \\
    \end{align*}
    By treating \(\E{\left( X \,\vert\, Y \right)}\) as a random variable of \(Y\), then
    we can calculate its expected value such that
    \begin{equation*}
        \E{\left( X \right)} = \E{\left( \E{\left( X \,\vert\, Y \right)} \right)}.
    \end{equation*}
    This is known as the law of total expectation.
    \subsection{Expectation}
    The following property holds for both dependent and independent random variables \(X\) and \(Y\)
    \begin{equation*}
        \E{\left( X \pm Y \right)} = \E{\left( X \right)} \pm \E{\left( Y \right)}
    \end{equation*}
    If \(X\) and \(Y\) are independent then
    \begin{equation*}
        \E{\left( XY \right)} = \E{\left( X \right)} \E{\left( Y \right)}
    \end{equation*}
    \subsection{Variance of Independent Random Variables}
    If \(X\) and \(Y\) are independent then
    \begin{align*}
        \Var{\left( X \pm Y \right)} & = \Var{\left( X \right)} + \Var{\left( Y \right)}                                                                                               \\
        \Var{\left( XY \right)}      & = \Var{\left( X \right)} \Var{\left( Y \right)} + \E{\left( X \right)}^2 \Var{\left( Y \right)} + \E{\left( Y \right)}^2 \Var{\left( X \right)}
    \end{align*}
    \subsection{Covariance}
    \subsection{Covariance}
    Covariance is a measure of the dependence between two random variables, it can be determined using
    \begin{align*}
        \Cov{\left( X,\: Y \right)} & = \E{\left( \left( X - \E{\left( X \right)} \right) \left( Y - \E{\left( Y \right)} \right) \right)} \\
                                    & = \E{\left( XY \right)} - \E{\left( X \right)} \E{\left( Y \right)}
    \end{align*}
    The covariance of \(X\) and \(Y\) is:
    \begin{description}
        \item[Positive] if an increase in one variable is more likely to result in an increase in
            the other variable.
        \item[Negative] if an increase in one variable is more likely to result in a decrease in
            the other variable.
        \item[Zero] if \(X\) and \(Y\) are independent. Note that the converse is not true.
    \end{description}
    The linear transformation of two random variables have the following covariance
    \begin{equation*}
        \Cov{\left( aX + b,\: cY + d \right)} = ac \Cov{\left( X,\: Y \right)}
    \end{equation*}
    for constants \(a\), \(b\), \(c\), and \(d\).
    \subsection{Joint expectation}
    The joint expectation of two discrete random variables is
    \begin{equation*}
        \E{\left( XY \right)} = \sum_{x\in\Omega_1} \sum_{y\in\Omega_2} xy \Pr{\left( X = x,\: Y = y \right)}
    \end{equation*}
    and for continuous random variables
    \begin{equation*}
        \E{\left( XY \right)} = \int_{x_1}^{x_2} \int_{x_1}^{x_2} xy f\left( x,\: y \right) \odif{y} \odif{x}.
    \end{equation*}
    \subsection{Variance of Dependent Random Variables}
    If \(X\) and \(Y\) are dependent then
    \begin{equation*}
        \Var{\left( X \pm Y \right)} = \Var{\left( X \right)} + \Var{\left( Y \right)} \pm 2\Cov{\left( X,\: Y \right)}
    \end{equation*}
    \subsection{Correlation}
    The covariance of two random variables describes the direction of a relationship, however
    it does not quantify the strength of such a relationship. The correlation explains both
    the direction and strength of a linear relationship between two random variables.

    The correlation of two random variables \(X\) and \(Y\) is denoted \(\rho\left( X,\: Y \right)\)
    \begin{equation*}
        \rho\left( X,\: Y \right) = \frac{\Cov{\left( X,\: Y \right)}}{\sqrt{\Var{\left( X \right)} \Var{\left( Y \right)}}}
    \end{equation*}
    where \(-1 \leq \rho\left( X,\: Y \right) \leq 1\).

    These value can be interpretted as follows:
    \begin{itemize}
        \item \(\rho\left( X,\: Y \right) > 0\) iff \(X\) and \(Y\) have a positive linear relationship.
        \item \(\rho\left( X,\: Y \right) < 0\) iff \(X\) and \(Y\) have a negative linear relationship.
        \item \(\rho\left( X,\: Y \right) = 0\) if \(X\) and \(Y\) are independent. Note that the converse is not true.
        \item \(\rho\left( X,\: Y \right) = 1\) iff \(X\) and \(Y\) have a perfect linear relationship with positive slope.
        \item \(\rho\left( X,\: Y \right) = -1\) iff \(X\) and \(Y\) have a perfect linear relationship with negative slope.
    \end{itemize}
    Note that the slope of a perfect linear relationship cannot be obtained from the correlation.
\end{multicols}

\end{document}